% Created 2026-01-14 Wed 15:40
% Intended LaTeX compiler: pdflatex
\documentclass[12pt]{article}
\usepackage[utf8]{inputenc}
\usepackage{titletoc}
\usepackage{graphicx} %Enhanced support for grpahics
\usepackage{tabularx} %Tables
\usepackage{fullpage}
\usepackage{grffile} %File name support for graphics
\usepackage{longtable}
\usepackage{wrapfig}
\usepackage{listingsutf8} %UTF8 file conversion?  \usepackage[style=numeric,doi=false,isbn=false,firstinits=true,sorting=none,backend=biber]{biblatex}
\pagenumbering{arabic}
  
\usepackage[utf8]{inputenc}
\usepackage[T1]{fontenc}
\usepackage{graphicx}
\usepackage{longtable}
\usepackage{wrapfig}
\usepackage{rotating}
\usepackage[normalem]{ulem}
\usepackage{amsmath}
\usepackage{amssymb}
\usepackage{capt-of}
\usepackage{hyperref}
\usepackage{listingsutf8}
\usepackage{color}
\usepackage{listings}
\author{Jeffrey Szymanski}
\date{\today}
\title{Bioinformatics metadata}
\begin{document}

\maketitle
\begin{itemize}
\item Elements\\[0pt]
\begin{itemize}
\item Immuntable reference files\\[0pt]
\begin{itemize}
\item Possible PHI\\[0pt]
\item Provenence\\[0pt]
\end{itemize}
\item Mutable "sample sheet"\\[0pt]
\begin{itemize}
\item xlsx or series of tsvs\\[0pt]
\item No PHI\\[0pt]
\item Source of truth\\[0pt]
\item Validation input\\[0pt]
\end{itemize}
\end{itemize}
\item No PHI\\[0pt]
\item With Libreoffice calc\\[0pt]
\begin{itemize}
\item Use Excel A1 formula syntax (Tools -> Options -> LibreOffice Calc -> Formula -> Formula Options -> Formula syntax)\\[0pt]
\end{itemize}
\end{itemize}
\section*{Logical data representation, Common Data Model schema}
\label{sec:org4a0e532}
\subsection*{Documentation}
\label{sec:orgdb0fe54}
Good Best Practices for Bioinformatics Metadata Schema with Frictionless\\[0pt]

\begin{itemize}
\item Components\\[0pt]
\begin{itemize}
\item Excel as the user interface: non-programmers enter data, schema validates it. Hierarchical data model: organize by biological and technical relationships\\[0pt]
\item schema yaml\\[0pt]
Self-documenting: schemas generate data dictionaries programmatically\\[0pt]
\end{itemize}
\item Schema Structure\\[0pt]
\begin{itemize}
\item Version control by repo-level versioning\\[0pt]
\item Define Shared Elements Once, Use YAML anchors for repeated definitions:\\[0pt]
\item Basic Resource Structure\\[0pt]
\item Field Definition Principles\\[0pt]
\begin{itemize}
\item Every field should include:\\[0pt]
\begin{itemize}
\item name: Machine-readable identifier\\[0pt]
\item title: Human-readable label\\[0pt]
\item description: Clear explanation including format requirements and units\\[0pt]
\item example: Representative value\\[0pt]
\item type: Data type\\[0pt]
\item constraints: Validation rules\\[0pt]
\item Example:\\[0pt]
\end{itemize}
\end{itemize}
\item Primary Key Conventions\\[0pt]
\begin{itemize}
\item Use descriptive prefix plus zero-padded four-digit numbers:\\[0pt]
\begin{itemize}
\item Enables lexicographic sorting\\[0pt]
\item Format: \^{}[prefix]\d{4}\$\\[0pt]
\item Examples: subj0001, samp0001, lib0001, run0001\\[0pt]
\item Always include pattern constraint to catch typos.\\[0pt]
\end{itemize}
\end{itemize}
\item Foreign Key Conventions\\[0pt]
\begin{itemize}
\item Foreign key fields should be minimal:\\[0pt]
\item Include pattern constraint to validate format before checking referential integrity.\\[0pt]
\item Detailed metadata lives in the parent table only - avoid redundancy.\\[0pt]
\end{itemize}
\item Data Type Decision Principles\\[0pt]
\begin{itemize}
\item Prefer Enums Over Booleans. Use enums for domain-specific concepts that require explanation:\\[0pt]
\item Booleans are acceptable only when semantics are universally obvious and require no explanation.\\[0pt]
\item Document Enum Values\\[0pt]
\begin{itemize}
\item Use custom enumDescriptions property for programmatic data dictionary generation:\\[0pt]
\end{itemize}
\item Standardize Units in Field Names: Include units in field name and description:\\[0pt]
\end{itemize}
\end{itemize}
\item Table Design Principles\\[0pt]
\begin{itemize}
\item Hierarchical Relationships\\[0pt]
\begin{itemize}
\item Organize tables by biological and technical hierarchy:\\[0pt]
\begin{verbatim}
subjects → samples → libraries → fastqs
\end{verbatim}
\item Use foreign keys to enforce referential integrity.\\[0pt]
\end{itemize}
\item When to Split Tables\\[0pt]
\begin{itemize}
\item Split into separate tables when entities have:\\[0pt]
\begin{itemize}
\item Many divergent fields (>5 unique per subtype)\\[0pt]
\item Fundamentally different semantics\\[0pt]
\item Example: human participants vs mouse subjects\\[0pt]
\end{itemize}
\end{itemize}
\item Keep entities in one table when:\\[0pt]
\begin{itemize}
\item Few optional fields (2-3)\\[0pt]
\item Structural similarity between subtypes\\[0pt]
\item Add discriminator field for subtype and leave subtype-specific fields optional:\\[0pt]
\end{itemize}
\end{itemize}
\item Schema Evolution Principles\\[0pt]
\begin{itemize}
\item Never use "other" catch-all categories or allow arbitrary values - this defeats validation.\\[0pt]
\item Constraint Defaults\\[0pt]
\begin{itemize}
\item Frictionless defaults are:\\[0pt]
\begin{itemize}
\item required: false (field is optional)\\[0pt]
\item unique: false (duplicates allowed)\\[0pt]
\item No enum restrictions\\[0pt]
\end{itemize}
\end{itemize}
\item Only declare constraints when restricting behavior.\\[0pt]
\end{itemize}
\item Field Ordering\\[0pt]
\begin{itemize}
\item Primary key first\\[0pt]
\item Foreign keys second\\[0pt]
\item Required fields\\[0pt]
\item Optional fields\\[0pt]
\item Group related fields together\\[0pt]
\end{itemize}
\end{itemize}

\subsection*{Preamble}
\label{sec:org7aab868}
\subsection*{Subjects}
\label{sec:orgcf94344}
\subsection*{Participants}
\label{sec:orgc173427}
\subsection*{Mice}
\label{sec:orgc80a530}
\subsection*{Samples}
\label{sec:org438f2d3}
\subsection*{Libraries}
\label{sec:orge6410e9}
\subsection*{Sequencing runs}
\label{sec:org42fde55}
\subsection*{FASTQs}
\label{sec:org168a8ef}
\#+end\_src
\end{document}
